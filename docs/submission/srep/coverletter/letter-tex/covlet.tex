\documentclass[10pt]{letter}
\usepackage{fontawesome}
\usepackage{hyperref}

% Margins
\topmargin=-1in % Moves the top of the document 1 inch above the default
\textheight=8.5in % Total height of the text on the page before text goes on to the next page, this can be increased in a longer letter
\oddsidemargin=-10pt % Position of the left margin, can be negative or positive if you want more or less room
\textwidth=6.5in % Total width of the text, increase this if the left margin was decreased and vice-versa
%\let\raggedleft\raggedright % Pushes the date (at the top) to the left, comment this line to have the date on the right

\begin{document}
%----------------------------------------------------------------------------------------
%	ADDRESSEE SECTION
%----------------------------------------------------------------------------------------
\begin{letter}{Dr. Richard White \\
Chief Editor of Scientific Reports \\
Springer Nature Limited \\
The Campus, 4 Crinan Street\\
London, N1 9XW \\
UK
} 

%----------------------------------------------------------------------------------------
%	YOUR NAME & ADDRESS SECTION
%----------------------------------------------------------------------------------------
\begin{flushright}
\large\bf Miguel Xochicale \\ % Your name
%\vspace{20pt} \hrule height 1pt % If you would like a horizontal line separating the name from the address, uncomment the line to the left of this text
Edgbaston\\
Birmingham\\
B15 2TT\\
UK\\
\faEnvelopeO  perez.xochicale@gmail.com \\
\faMobile  (+44) 0744 281 7616 \\
\faHome \href{http://mxochicale.github.io}{http://mxochicale.github.io}
%\vspace{20pt}
%\large\bf Chris Baber, Professor \\ % Your name
%%\vspace{20pt} \hrule height 1pt % If you would like a horizontal line separating the name from the address, uncomment the line to the left of this text
\end{flushright} 

%\vfill
\vspace{20pt}
%\author{Miguel Xochicale, King's College London}
\signature{Miguel Xochicale} % Your name for the signature at the bottom

%----------------------------------------------------------------------------------------
%	LETTER CONTENT SECTION
%----------------------------------------------------------------------------------------

\opening{Dear Editor Richard White,}

%* The affiliation and contact information of your corresponding author
I have the pleasure of sending you the manuscript entitled "Nonlinear methods to quantify Movement Variability in Human-Humanoid Interaction Activities" authored by Miguel Xochicale and Chris Baber from the University of Birmingham to be considered for publication as a research article in your prestigious journal.

%* A brief explanation of why the work is appropriate for Scientific Reports
Our work is appropriate for Scientific Reports due to its alignment to the principles of open science (e.g., reproducibility, transparency, reusability and open accessibility) in the fields of physical, biological, health and engineering sciences.
Similarly, we choose Scientific Reports as venue where potential interest of readers in above mentioned areas along with raising interested and spark collaborations.

Our manuscript contains original research and has not been submitted or published earlier in any journal and is not being considered for publication elsewhere.   
Authors have seen and approved the manuscript and have contributed significantly to it.

%* The names and contact information of any reviewers you consider suitable
%* The names of any referees you would like excluded from reviewing
Finally, we think that two suitable reviewers for our work would be: 
Bedartha Goswami, bedartha.goswami@uni-tuebingen.de, at University of Tübingen, for his investigation in nonlinear time series analysis and recurrence plots; 
and Serhiy Yanchuk, yanchuk@math.tu-berlin.de, at Technical University of Berlin, for his research in dynamics time delay networks of dynamical systems, pattern formation, and coupled systems.  

\closing{Sincerely yours,}

% List your enclosed documents here, comment this out to get rid of the "encl:"
\encl{Manuscript and supplementary document} 

\end{letter}
\end{document}
