\documentclass[10pt]{article}
\usepackage[margin=15mm]{geometry}
\usepackage{enumitem}
\usepackage{hyperref}

\title{
Logbook of corrections 2020-01-14 
}
\author{Miguel Xochicale}
\date{ \today }



\begin{document}
\maketitle

\begin{abstract}
This logbook contains comments made by Chris Baber 
(\emph{.../corrections/2020-01-14/comments/README.md})
and their responses. 
However, on 4th of February 2020, I realised 
that before tackling such comments (in Section I),
I will be reviewing the whole manuscript and report 
any amends (this start in section II).
\end{abstract}

\tableofcontents

%%%%%%%%%%%%%%%%%%%%%%%%%%%%%%%%%%%%%%%%%%%%%%%%%%%%%%%%%%%%%%
\section{Chris Baber}
\subsection{Summarise results}


\begin{enumerate}


\item 

At the end of the results section (before the Discussion) you should include a paragraph that summarises the main results
	\begin{verbatim}


Tue 30 Jun 00:21:09 BST 2020
To summarise this section of results, it has been found that computing individual embedding parameters 
for a particular structure of time series is an already solved problem
\cite{frank2010, sama2013, bradley2015}.
However, the challenge is to find embedding parameters that represent a set of different time series,
in order to compute uniform time-delay embedding, recurrence plots and recurrence quantification analysis.
That said, it has been proposed the use of sample mean of the set of embedding parameters
to then noticed that the selection of recurrence threshold, $\epsilon$, is also an open problem.
Hence, variation of recurrence thresholds and embedding parameters 
has been presented to show the relationships to different datasets 
(participants, activities, windows lengths and sensors).



Thu  2 Jul 20:57:04 BST 2020
To summarise this section of results, it can be said that computing
embedding parameters for individual structure of time-series 
data is already a solved problem \cite{frank2010, sama2013, bradley2015}. 
However, it has been shown the challenge of finding embedding parameters 
for nonlinear dynamic tools that represent a set of different time-series data.
That said, we proposed the use of sample mean of the set of embedding parameters
for RSSs, RP and RQA to then noticed that the selection of recurrence 
threshold, $\epsilon$, is also an open problem.
For which, this work proposed the variation of recurrence thresholds 
and embedding parameters to show the relationships of these to different datasets 
(participants, activities, windows lengths and sensors).


	\end{verbatim}
	\textit{
	SORTED: \\ 
Tue 30 Jun 00:21:15 BST 2020 \\
Thu  2 Jul 20:57:15 BST 2020\\
	}
	\\


\end{enumerate}



%%%%%%%%%%%%%%%%%%%%%%%%%%%%%%%%%%%%%%%%%%%%%%%%%%%%%%%%%%%%%%
\subsection{Comments for the conclusion}


\begin{enumerate}

\item 

It is still not clear what conclusions you can draw about the 
performance of the participants (in the experiment) or 
comparing human with humanoid. While I can see that the point 
of the paper is on the discussion of the different methods, 
and their usefulness, 
I still think it would be good 
to make observations on variability in the actual 
movement (what is the difference between human and humanoid?, 
do some people show more variability than others?) 
and then ask whether (as well as being sensitive to the data) 
some of the methods are more sensitive to variability 
in performance.
	\begin{verbatim}




Thu  2 Jul 22:41:43 BST 2020
\section*{Conclusions}
This work allows to conclude that the choice of 
nonlinear analysis tool (e.g., RSSs, RPs, RQA metrics) will depend 
on what one would like to quantify on the time-series 
data (e.g., predicability, organisation, dynamics transitions, 
or complexity and determinism).
Then, time-series data characteristics 
(e.g., window size length, level of smoothness) plays an important role
as well on the results that nonlinear analysis tool can provide.
Similarly, the results of the nonlinear analysis tools
are associated with the structure of the time-series data 
(e.g. frequency, amplitude), the position of the sensor 
and activity performed by either a robot or human being
(as degrees of freedom from the humanoid are far less than
human movement). 
That said, it has been shown that the use of different 
characteristics of the time-series data 
(e.g., sensor, activity, level of smoothness and participants)
has help us to visualise and to quantify 
with nonlinear tools the variation of 
movements of, in this work, human-humanoid activities.
However, some limitation of nonlinear tools are related 
to the computation of different parameters 
(e.g., recurrence thresholds, embedding parameters)
that reflect the dynamics of individual characteristics 
of activity type, window length and structure of the time series.
Specifically, the example of DET values which appear 
to be constants across sensors, activities and levels of smoothness, 
whereas REC and RATIO, as function of REC, values 
show variation for certain sensors and movements. 
To then found out that RQA ENTR values with different recurrence 
thresholds were appropriate to quantify 
the different changes and variations of the characteristics of 
time-series data.
Therefore, we can conclude that this work provides a good starting 
point and reference to the use of Shannon Entropy to quantify 
human-humanoid imitation activities 
that can then lead to interesting results on the quantification 
of movement variability of participants with different ages, 
state of health and anthropomorphic features.


	
	\end{verbatim}
	\textit{
	LOG: \\
	Thu  2 Jul 22:41:43 BST 2020 \\
	}
	\\


\end{enumerate}



%%%%%%%%%%%%%%%%%%%%%%%%%%%%%%%%%%%%%%%%%%%%%%%%%%%%%%%%%%%%%%%
%\subsection{Comments and question 2}
%
%
%\begin{enumerate}
%
%\item 
%
%It is still not clear what conclusions you can draw about the 
%performance of the participants (in the experiment) or 
%comparing human with humanoid. While I can see that the point 
%of the paper is on the discussion of the different methods, 
%and their usefulness, 
%I still think it would be good 
%to make observations on variability in the actual 
%movement (what is the difference between human and humanoid?, 
%\textbf{do some people show more variability than others?)}
%and then ask whether (as well as being sensitive to the data) 
%some of the methods are more sensitive to variability 
%in performance.
%
%	\begin{verbatim}
%
%	
%	\end{verbatim}
%	\textit{
%	SORTED: 
%	}
%	\\
%
%
%\end{enumerate}
%
%
%%%%%%%%%%%%%%%%%%%%%%%%%%%%%%%%%%%%%%%%%%%%%%%%%%%%%%%%%%%%%%%
%\subsection{Comments and question 3}
%
%
%\begin{enumerate}
%
%\item 
%
%It is still not clear what conclusions you can draw about the 
%performance of the participants (in the experiment) or 
%comparing human with humanoid. While I can see that the point 
%of the paper is on the discussion of the different methods, 
%and their usefulness, 
%I still think it would be good 
%to make observations on variability in the actual 
%movement (what is the difference between human and humanoid?, 
%do some people show more variability than others?) 
%and 
%\textbf{
%then ask whether (as well as being sensitive to the data) 
%some of the methods are more sensitive to variability 
%in performance.}
%
%	\begin{verbatim}
%
%	
%	\end{verbatim}
%	\textit{
%	SORTED: 
%	}
%	\\
%
%
%\end{enumerate}
%
%




\newpage

%%%%%%%%%%%%%%%%%%%%%%%%%%%%%%%
\section{Miguel Xochicale}
\subsection{Introduction}
\begin{enumerate}

\item  (pp. ) 
	Human movement requires a complex system where not only multiple
joints and limbs are involved for a specific task in a determined environment
but also perception and action of movement affects the physical performance (davids2003)




	\begin{verbatim}
Human movement requires a complex system where not only multiple
joints and limbs are involved for a specific task in a determined environment
but also perception and action of movement that affects such 
physical performance 
\cite{davids2003}. 

	\end{verbatim}
	\textit{
	ADDED: Mon 10 Feb 07:08:01 GMT 2020
	}
	\\


\item  (pp. ) 
In contrast, variability in humanoid movement is usually very small,
as a result of its control system cite{gouaillier2009}. 

 
	\begin{verbatim}
In contrast, variability in humanoid movement is usually very small,
as a result of mechanical and dynamic capabilities \cite{gouaillier2009}. 
\end{verbatim}
	\textit{
	ADDED: Mon 10 Feb 07:25:54 GMT 2020
	}
	\\


\item  (pp. ) 
humans tend to have more fluid and flexible approach. Consequently, 
one can see much variability in performance of 
even the simplest task.


\begin{verbatim}
humans tend to have more fluid and flexible approach. Consequently, 
one can see much variability in human movement performance of 
even the simplest task.
\end{verbatim}
	\textit{
	ADDED: Mon 10 Feb 07:30:31 GMT 2020
	}
	\\


\item  (pp. ) 
nonlinear time series analysis based on the appropriate estimation of the 
\begin{verbatim}
nonlinear time series analysis based on the estimation of the 
\end{verbatim}
	\textit{
	ADDED: Mon 10 Feb 16:32:51 GMT 2020
	}
	\\


\item  (pp. ) 
Bradley et al. 2015 cite{bradley2015} also reviewed the use of 
Recurrence Plots (RPs), a graphical representation of a two-dimensional map 
which show black and white dots as recurrences in a given $n$-dimensional system, 
and Recurrence Quantification Analysis (RQA) metrics compute statistics in RPs.


\begin{verbatim}

Similarly, Bradley et al. 2015 \cite{bradley2015} reviewed the use of 
Recurrence Plots (RPs), a graphical representation of a two-dimensional map 
which show black and white dots as recurrences in a given $n$-dimensional system, 
and Recurrence Quantification Analysis (RQA) metrics that compute 
statistics in RPs.


\end{verbatim}
	\textit{
	ADDED: Mon 10 Feb 16:42:27 GMT 2020
	}
	\\




\item  (pp. ) 
Additionally, the methodologies of nonlinear dynamics for 
computing the embedding parameters e.g., autocorrelation, mutual information, 
and nearest neighbour require data which is well 
sampled and with little noise \ cite{garland2016}
or require purely deterministic signals \ cite{kantz2003}.

\begin{verbatim}
Additionally, methods of nonlinear dynamics for 
computing embedding parameters e.g., autocorrelation, mutual information, 
and nearest neighbour require data which is well 
sampled and with little noise \cite{garland2016}
or require signals that are purely deterministic \cite{kantz2003}.

\end{verbatim}
	\textit{
	AMENDED: Mon 10 Feb 16:55:30 GMT 2020
	}
	\\



\item  (pp. ) 

Thus, these nonlinear analysis methods for computing the embedding 
parameters can break down
with real-world datasets which have generally different length, 
different values of accuracy and precision (rounding errors due to finite 
precision of the measurement apparatus which include frequency 
acquisition \ cite{frank2010}),
and data may be contaminated with different sources of noise
\ cite{garland2016}.

\begin{verbatim}
Thus, these methods of nonlinear analysis for computing the embedding 
parameters can break down
with real-world datasets which have generally different length, 
different values of accuracy and precision (rounding errors due to finite 
precision of the measurement apparatus which include frequency 
acquisition \cite{frank2010}),
and contaminated data with different sources of noise
\cite{garland2016}.
\end{verbatim}
	\textit{
	AMENDED: 
Mon 10 Feb 17:00:32 GMT 2020
	}
	\\


\item  (pp. ) 
It is surprising that even with the previous constraints with regard 
to the quality of data, and the problem with the estimation of embedding parameters,
the use of nonlinear dynamics have proven to be helpful to understand and 
characterise dynamics of time series 
\
\begin{verbatim}
It is then surprising that even with the previous constraints with regard 
to the quality of data, and the problem with the estimation of 
embedding parameters,
the use of nonlinear dynamics have proven to be helpful to understand and 
to characterise dynamics of time series 
\
\end{verbatim}
	\textit{
	AMENDED: Mon 10 Feb 17:02:59 GMT 2020
	}
	\\



\end{enumerate}



%%%%%%%%%%%%%%%%%%%%%%%%%%%%%%%
\subsection{Results}
\begin{enumerate}


\item  (pp. ) 
One of the challenges of the implementation of 
Uniform Time-Delay Embedding to Reconstructed State Spaces is the 
selection of embedding parameters because each time series is unique in 
terms of its structure (modulation of amplitude, frequency and phase) 
\ cite{ frank2010, sama2013, bradley2015}.

\begin{verbatim}
One of the challenges of the implementation of 
uniform time-delay embedding to reconstruct the state spaces is the 
selection of embedding parameters because each time series is unique in 
terms of its structure (e.g. modulation of amplitude, frequency and phase) 
\cite{ frank2010, sama2013, bradley2015}.
\end{verbatim}
	\textit{
	AMENDED: 
Tue 11 Feb 07:35:10 GMT 2020
	}
	\\



\item  (pp. ) 
With that in mind, the problem for this work is not to compute individual 
embedding parameters for each of the time series but to deal with a 
selecting of two parameters that can represent all the time series. 
Our solution for that problem was to compute a sample mean over all 
values in each of the conditions of the time series 
%of Figs~\ref{fig:ts}
for minimum dimension values and 
%Figs~\ref{fig:caoami} 
for minimum delay values as follows.
\begin{verbatim}
With that in mind, the problem for this work is not to compute individual 
embedding parameters for each of the time series but to deal with a 
selection of two parameters that can represent all the time series. 
Our solution for such problem, as explained in below sections, 
was to compute a sample mean over all 
values in each of the conditions of the time series 
%of Figs~\ref{fig:ts}
for minimum dimension values and 
%Figs~\ref{fig:caoami} 
for minimum delay values.
%
\end{verbatim}
	\textit{
	AMENDED: 
Tue 11 Feb 03:47:57 GMT 2020
	}
	\\



\item  (pp. ) 
Minimum embedding parameters were firstly computed 
with False Nearest Neighbour and Average Mutual Information.
%
\begin{verbatim}
Minimum embedding parameters were firstly computed 
with the methods of False Nearest Neighbour and Average Mutual Information.
%
\end{verbatim}
	\textit{
	AMENDED: 
Wed 12 Feb 10:30:37 GMT 2020
	}
	\\

\item  (pp. ) 

	{\bf Box plots of minimum embedding parameters for horizontal and vertical arm movements.} 

	Minimum embedding dimensions are for twenty participants 
	($p01$ to $p20$) with three smoothed signals 
	(sg0: sg0zmuvGyroZ, sg1: sg1zmuvGyroZ and sg2: sg2zmuvGyroZ)
	and window length of 10-sec (500 samples).
	Code and data to reproduce the figure is available from \cite{srep2019}.
        

\begin{verbatim}

	{\bf Box plots of minimum embedding parameters.}


	Minimum embedding parameters are for twenty participants 
	($p01$ to $p20$) with three smoothed signals 
	(sg0: sg0zmuvGyroZ, sg1: sg1zmuvGyroZ and sg2: sg2zmuvGyroZ)
	and window length of 10-sec (500 samples).
	Code and data to reproduce the figure is available in \cite{srep2019}.
        }

\end{verbatim}
	\textit{
	AMENDED: \\ 
Thu  5 Mar 06:46:20 GMT 2020 \\
Sat  7 Mar 15:35:36 GMT 2020
	}
	\\



\item  UPDATE REFERENCE AND AMEND LEGEND IN ALL FIGURES \\

	Code and data to reproduce the figure is available from \textbackslash cite{srep2019}.

\begin{verbatim}

	Code and data to reproduce the figure is available in \cite{srep2020}.


\end{verbatim}
	\textit{
	AMENDED: 
Tue 10 Mar 23:46:25 GMT 2020
	}
	\\



\item  Change title 

\title{The Application of Recurrent Quantification Analysis 
to Human-Humanoid Interaction}

\begin{verbatim}

\title{Recurrent Quantification Analysis of Human-Humanoid Interaction Activities}

\end{verbatim}
	\textit{
	AMENDED: 
Tue 10 Mar 23:42:40 GMT 2020
	}
	\\




\item  (pp. ) 

ORIGINAL
\begin{verbatim}
Figures \ref{fig:cao_ami}(A) show that minimum embedding values 
for sensor HS01 appear to show more variations as 
interquartile range is near to 1 
with three exceptions (HF sg0, VN sg0, and VF sg1).
Minimum embedding values appear to be constant for sensor RS01 
as interquartile range is near to 0.1 with the exception of 
two axis (HF sg0 and VF sg1).
Additionally, it can be seen in Figs \ref{fig:cao_ami}(A) a 
decrease of mean values (rhombus) in the box plots
as smoothness of time series increase.
Similarly, the first minimum values of AMI values are 
shown in the box plots of Figs \ref{fig:cao_ami}(B).
It can be seen that values for HS01 tend to be more spread as the smoothness 
of the time series is increasing 
(see the increase of both mean (rhombus) and interquartile range).
However, AMI values for RS01 do not show such increase in relation with
the increase of smoothness except for HF and VF
(Figs \ref{fig:caoami}(B)).

We also computed an overall minimum embedding parameters that represent 
participants, activities, sensors and levels of smoothness, using a sample 
mean of all values in Figs~\ref{fig:cao_ami}.
The sample mean for the minimum values of $E_{1}(m)$ from Figs \ref{fig:cao_ami}(A) 
is $\overline{m}_0=6$ 
and the sample mean for minimum values of AMIs from Figs~\ref{fig:cao_ami}(B)
is $\overline{\tau}_0=8$
\end{verbatim}
	

AMENDED:
\begin{verbatim}
Figures \ref{fig:cao_ami}(A) show box plots for the minimum embedding values 
for sensors HS01 and RS01. Minimum embedding vaues for HS01 presents more 
variations than RS01 as interquartile range for HS01 is near to 1 (with the exceptions of HF sg0, VN sg0, and VF sg1)
whereas interquartile range for RS01 is near to 0.1 (with the exception of two axis HF sg0 and VF sg1).
Additionally, Figures \ref{fig:cao_ami}(A) show
a decrease of mean values (rhombus) in the box plots
as smoothness of time series increase (see sg0, sg1, sg2).
Figures \ref{fig:cao_ami}(B) show box plots for the first minimum values of AMI values.
It can be seen that values for HS01 tend to be more spread as the smoothness 
of the time series is increasing 
(see the increase of both mean (rhombus) and interquartile range).
However, AMI values for RS01 do not show such increase in relation with
the increase of smoothness except for HF and VF
(Figs \ref{fig:cao_ami}(B)).

We also computed a sample mean for an overall minimum embedding parameters that 
represent all participants, activities, sensors and levels of smoothness.
The sample mean for the minimum values of $E_{1}(m)$ from Figs \ref{fig:cao_ami}(A) 
is $\overline{m}_0=6$ 
and the sample mean for minimum values of AMIs from Figs~\ref{fig:cao_ami}(B)
is $\overline{\tau}_0=8$.
%
\end{verbatim}

\textit{
	LOG: 
Thu 19 Mar 21:58:23 GMT 2020
	}
	\\




\item  (pp. ) 

ORIGINAL
\begin{verbatim}
Uniform time-delay embedding were computed with
embedding parameters ($\overline{m_0}=6$, $\overline{\tau_0}=8$) and 
the first three axis of the rotated data of the PCA are shown 
for the reconstructed state spaces in 
Figs~\ref{fig:rss_aHw10} for horizontal arm movements and 
Figs~\ref{fig:rss_aVw10} for vertical arm movements.

It is easy to observe by eye the differences in each of the
trajectories in the reconstructed state spaces 
(Figs~\ref{fig:rss_aHw10}, \ref{fig:rss_aVw10}), 
however one might be not objective when quantifying those differences 
since those observation might vary from person to person.
With that in mind, we tried to objectively quantify those differences 
using euclidean distances between the origin to each of the points in the 
trajectories, however these results created a suspicious metric, specially 
for trajectories which looked very messy.
With that in mind, we computed Recurrence Plots and 
Recurrence Quantification Analysis to objectively quantify 
the differences in each of the cases of the time series.

\end{verbatim}
AMENDED
\begin{verbatim}

Uniform time-delay embedding were computed with the overall 
embedding parameters ($\overline{m_0}=6$, $\overline{\tau_0}=8$) and 
the first three axis of the rotated data of the PCA are shown 
for the reconstructed state spaces of horizontal arm movements
(Figs~\ref{fig:rss_aHw10}) and vertical arm movements 
(Figs~\ref{fig:rss_aVw10}).
%
One can observe by eye the differences in each of the
trajectories of the reconstructed state spaces 
(Figs~\ref{fig:rss_aHw10} and \ref{fig:rss_aVw10}), 
however an objective quantification is required.
That being said, we tried to quantify those differences 
using euclidean distances between the origin 
to each of the points in their trajectories, 
which results did not capture the dynamics of 
the trajectories specially 
for those that looked very messy.
Such ambiguities lead us to consider the use of 
Recurrence Plots and Recurrence Quantification Analysis 
to objectively quantify dynamics the capture activities 
using time series.


\end{verbatim}

	\textit{
LOG: \\
Thu 19 Mar 22:21:44 GMT 2020  \\
Mon 13 Apr 13:22:42 BST 2020
	}
	\\




\item  (pp. ) 

ORIGINAL
\begin{verbatim}
Recurrence Plots (RP) were computed for horizontal arm movements (Fig~\ref{fig:rp_aH}) and 
vertical arm movements (Fig~\ref{fig:rp_aV}) 
using the average embedding parameters ($m=6$, $\tau=8$) 
and an recurrence threshold of $\epsilon=1$.

	embedding parameters $m=6$, $\tau=8$ and $\epsilon=1$.
\end{verbatim}

AMENDED
\begin{verbatim}
Recurrence Plots (RP) were computed for horizontal arm movements (Fig~\ref{fig:rp_aH}) and 
vertical arm movements (Fig~\ref{fig:rp_aV}) 
using the average embedding parameters ($m=6$, $\tau=8$) 
and an recurrence threshold of $\epsilon=1$.


	$\overline{m}_0=6$, $\overline{\tau}_0=8$, and $\epsilon=1$.
\end{verbatim}

	\textit{
LOG: \\
 Mon 13 Apr 13:31:49 BST 2020
	}
	\\




\item  (pp. ) \\ 
ORIGINAL
\begin{verbatim}

Similarly to RP, we computed four RQA metrics (REC, DET, RATIO and ENTR) 
with embedding parameters $m=6$, $\tau=8$ and 
recurrence threshold $\epsilon=1$.
 

\end{verbatim}

AMENDED
\begin{verbatim}
Four Recurrence Quantification Analysis (RQA) metrics (REC, DET, RATIO and ENTR) 
were also computed with $\overline{m}_0=6$, $\overline{\tau}_0=8$, and $\epsilon=1$. 
 

\end{verbatim}

\textit{
LOG: \\ 
Mon 13 Apr 13:42:03 BST 2020
}
\\



\item  (pp. ) \\
ORIGINAL
\begin{verbatim}

	[Box plots for RQA values]{
	{\bf Box plots for RQA values.}
	Box plots of (A) REC, (B) DET, (C) RATIO, and (D) ENTR values 
	for 20 participants performing HN, HF, VN and VF movements
	with sensors HS01, RS01 and three smoothed-normalised  
	time series (sg0, sg1 and sg2).
	RQA values were computed with 
	embedding parameters $m=6$, $\tau=8$ and $\epsilon=1$.
	Code and data to reproduce the figure is available in \cite{srep2020}.
 

\end{verbatim}

AMENDED
\begin{verbatim}

	[Box plots for RQA values]{
	{\bf Box plots for RQA metrics.}
	RQA metrics for (A) REC, (B) DET, (C) RATIO, and (D) ENTR of 
	20 participants performing HN, HF, VN and VF movements
	with sensors HS01, RS01 and three smoothed-normalised  
	time series (sg0, sg1 and sg2).
	RQA values were computed with 
	$\overline{m}_0=6$, $\overline{\tau}_0=8$, and $\epsilon=1$. 
	Code and data to reproduce the figure is available in \cite{srep2020}.
 

\end{verbatim}

\textit{
LOG: \\ 
Mon 13 Apr 13:53:42 BST 2020
}
\\




\item  (pp. ) \\
ORIGINAL
\begin{verbatim}
 
\subsubsection*{REC values}
It can be seen in the box plots of Figs~\ref{fig:RQABP}(A) that REC values, 
representing the \% of black dots in the RPs, 
are more spread for HN and VN movements (higher interquartile range) 
than HF and VF movements (lower interquartile range) for HS01 sensor. 
In contrast, REC values for RS01 sensor present little variation 
(interquartile range of 0.01).
With regard to the increase of smoothness of time series 
(sg0, sg1 and sg2), REC values present little 
variation as the smoothness is increasing for time series from HS01 
(changes of mean values (rhombus)) while REC values are more affected with 
the smoothness for data from RS01 
(see the incremental changes of mean values (rhombus)).
%See Figs~\ref{fig:rec_aH} and \ref{fig:rec_aV} in appendix \ref{appendix:e:ep} 
%for more details about individual REC values for each participant.



\end{verbatim}

AMENDED
\begin{verbatim}
Figs~\ref{fig:RQABP}(A) show box plots of REC values 
that represent the \% of black dots in RPs.
It can be noted that REC values are more spread 
for HN and VN movements (higher interquartile range) than 
for HF and VF movements (lower interquartile range) for HS01 sensor. 
In contrast, REC values for RS01 sensor present little variation 
(interquartile range of 0.01).
Regarding the increase of smoothness for time series 
(sg0, sg1 and sg2), REC values present little 
variation as the smoothness is increasing for time series from HS01 
(changes of mean values (rhombus)) whereas REC values are 
more affected with the smoothness for data from RS01 
(see the incremental changes of mean values (rhombus)).
%See Figs~\ref{fig:rec_aH} and \ref{fig:rec_aV} in appendix \ref{appendix:e:ep} 
%for more details about individual REC values for each participant.


\end{verbatim}

\textit{
LOG: \\ 
Mon 13 Apr 17:06:01 BST 2020
}
\\




\item  (pp. ) \\
ORIGINAL
\begin{verbatim}

\subsubsection*{DET values}
Figs \ref{fig:RQABP}(B) illustrate DET values, 
representing predictability and organisation of the RPs, 
which change very little (interquartile range is around 0.1) 
for type of movement, level of smoothness or type of sensor.
There is also increase of DET values as the smoothness of the signal increase 
(see the incremental changes of mean values (rhombus)).
It can be noted that the interquartile range for faster movements
(HF and VF) with no smoothing (sg0) is lower than the other
levels of smoothness (sg1 and sg2).
%See Figs~\ref{fig:det_aH} and \ref{fig:det_aV} in appendix \ref{appendix:e:ep} 
%for more details about individual DET values for each participant.


\end{verbatim}

AMENDED
\begin{verbatim}

\subsubsection*{DET values}
Figs \ref{fig:RQABP}(B) illustrate DET values 
that represent the predictability and organisation of RPs.
Generally, it can be noted little change of DET values 
(interquartile range is around 0.1) 
for type of movement, type of sensor 
but the increase of DET values as the smoothness of 
the signal increase 
(see the incremental changes of mean values (rhombus)).
However, the interquartile range for faster movements
(HF and VF) with no smoothing (sg0) is lower than the other
levels of smoothness (sg1 and sg2).
%See Figs~\ref{fig:det_aH} and \ref{fig:det_aV} in appendix \ref{appendix:e:ep} 
%for more details about individual DET values for each participant.


\end{verbatim}

\textit{
LOG: \\ 
Mon 13 Apr 17:28:11 BST 2020
}
\\





\item  (pp. ) \\
ORIGINAL
\begin{verbatim}
\subsubsection*{RATIO values}
Figs \ref{fig:RQABP}(C) present RATIO values, representing dynamic transitions, 
for horizontal and vertical movements.
It can be seen that RATIO values for HS01 sensor vary less 
for HN movements (interquartile range around 2)
than HF movements (interquartile range around 5).
%which is a similar behaviour of RATIO values for oRS01 sensor.
For faster movements (HF and VF), it can be noted a decrease of 
RATIO values as the smoothness of the time series is increasing (rhombus).
For normal movements (HN and VN), the decrease of RATIO values
is less evident than faster movements (see rhombus).
%See Figs~\ref{fig:ratio_aH} and \ref{fig:ratio_aV} in appendix 
%\ref{appendix:e:ep} 
%for more details about individual RATIO values for each participant.



\end{verbatim}

AMENDED
\begin{verbatim}

\subsubsection*{RATIO values}
Figs \ref{fig:RQABP}(C) illustrate RATIO values that represent the 
dynamic transitions in RPs. 
Generally, it can be seen that RATIO values for HS01 sensor vary less 
for HN movements (interquartile range around 2)
than HF movements (interquartile range around 5).
For faster movements (HF and VF), it can be noted a decrease of 
the mean values the smoothness of the time series 
is increasing (rhombus). However, for normal movements (HN and VN), 
the decrease of mean values is less evident than 
faster movements (rhombus).
%See Figs~\ref{fig:ratio_aH} and \ref{fig:ratio_aV} in appendix 
%\ref{appendix:e:ep} 
%for more details about individual RATIO values for each participant.


\end{verbatim}

\textit{
LOG: \\ 
Mon 13 Apr 17:45:43 BST 2020
}
\\




\item  (pp. ) \\
ORIGINAL
\begin{verbatim}
\subsubsection*{ENTR values}
Figs \ref{fig:RQABP}(D) show ENTR values, Shannon entropy values 
which represent the complexity of 
the structure the time series, 
for type of movement, level of smoothness or type of sensor.
ENTR values for HS01 sensor show a higher variation  
(interquartile range around 0.5)
than ENTR values for RS01 sensor which appear 
to be more constant (interquartile range 0.1).
It can also be noted the increase of smoothness of time series 
also is presented with an increase of ENTR values (see gray rhombus).
%See Figs~\ref{fig:entr_aH} and \ref{fig:entr_aV} in appendix
%\ref{appendix:e:ep}
%for more details about individual ENTR values for each participant.



\end{verbatim}

AMENDED
\begin{verbatim}
\subsubsection*{ENTR values}
Figs \ref{fig:RQABP}(D) show ENTR values that represent 
Shannon entropy values in RPs. 
Generally, 
ENTR values for HS01 sensor show a higher variation  
(interquartile range around 0.5)
than ENTR values for RS01 sensor 
(interquartile range 0.1).
It can also be noted the increase of mean values 
for ENTR values as the smoothness of time series 
increase (rhombus) for type of movement, 
type of sensor and level of smoothness.  
%See Figs~\ref{fig:entr_aH} and \ref{fig:entr_aV} in appendix
%\ref{appendix:e:ep}
%for more details about individual ENTR values for each participant.


\end{verbatim}

\textit{
LOG: \\ 
Mon 13 Apr 18:00:49 BST 2020
}
\\





\item  (pp. ) \\
ORIGINAL
\begin{verbatim}
	RQA ENTR values 
	for different embedding parameters
	$[1,10]= \{ m \in \mathbb{R} | 0 \le m \le 10  \} $,
	$[1,10]= \{ \tau \in \mathbb{R} | 0 \le \tau \le 10  \} $
	with an increase of 1
	and different recurrence thresholds $\epsilon=1, 2, 3$.
	RQA ENTR values are for $p03$, sensor HS01, of a window size of 10-secs (500 samples).
	Code and data to reproduce the figure is available in \cite{srep2020}.
 
\end{verbatim}

AMENDED
\begin{verbatim}
	[3D RQA ENTR values]{
	{\bf 3D RQA ENTR values.}
	RQA ENTR values 
	for different embedding parameters
	$ \{ m \in \mathbb{R} | 0 \le m \le 10  \} $,
	$ \{ \tau \in \mathbb{R} | 0 \le \tau \le 10  \} $
	incrementing by one
	and three recurrence thresholds $\epsilon=1, 2, 3$.
	RQA ENTR values were computed with data from $p03$, sensor HS01, with 
	a window size of 10-secs (500 samples).
	Code and data to reproduce the figure is available in \cite{srep2020}.
        
\end{verbatim}

\textit{
LOG: \\ 
Tue 14 Apr 09:19:00 BST 2020 \\
Tue 14 Apr 11:04:36 BST 2020
}
\\




\item  (pp. ) \\
ORIGINAL
\begin{verbatim}

\subsection*{3D RQA ENTR}
Choosing an appropriate recurrence threshold is crucial to get 
meaningful representations in RP and RQA, however, as previously shown, 
one can selected the default recurrence threshold of $\epsilon=1$, 
as long as it is able to represent the dynamical transitions 
in each of the time series \cite{marwan2011}. %%Marwan et al. \cite{marwan2011} 
However, to show how that the selection of recurrence threshold affects RQA
values, 
we computed ENTR values of RQA metrics for different embedding parameters
$[1,10]= \{ m \in \mathbb{R} | 1 \le m \le 10  \} $,
$[1,10]= \{ \tau \in \mathbb{R} | 1 \le \tau \le 10  \} $
with an increase of 1
and different recurrence thresholds 
$[0.2,3]= \{ \epsilon \in \mathbb{R} | 0.2 \le \epsilon \le 3.0  \} $
with increments of 0.1.
For easily visualisation, Figs \ref{fig:RQA-IND} only show 3D RQA for 
three recurrence thresholds $\epsilon=1, 2, 3$ with three levels of 
smoothness (sg0, sg1, sg2). It can be noted in Figs \ref{fig:RQA-IND} 
that the increase of recurrence threshold is associated to the increase 
of ENTR values in any of the levels of smoothness (see values of the ENTR bar).
One can also see that the maximum values of ENTR 
for $\epsilon=1$ and sg0 are for embedding dimensions of 2 
and then as $\epsilon$ increases the maximum values of ENTR are for
an associated increase of in the embedding dimension.
It can also be noted that the increase of level of smoothness (from sg0 to sg2) 
is associated with both the increase of ENTR values and its smoothed 3D shape.




\end{verbatim}

AMENDED
\begin{verbatim}
\subsection*{3D RQA ENTR}
It has been shown that one can select a default recurrence 
threshold of $\epsilon=1$ to represent the dynamical transitions 
in each of the time series \cite{marwan2011}.
%Choosing an appropriate recurrence threshold is crucial so as to get 
%meaningful representations in RP and RQA, however, 
However, to show how that the selection of recurrence threshold affects RQA
values, 
we computed ENTR values of RQA metrics for different embedding parameters
$ \{ m \in \mathbb{R} | 1 \le m \le 10  \} $,
$  \{ \tau \in \mathbb{R} | 1 \le \tau \le 10  \} $ incrementing by one, 
three recurrence thresholds $\epsilon=1, 2, 3$.
and three levels of smoothness (sg0, sg1, sg2). 
%different recurrence thresholds 
%$\{ \epsilon \in \mathbb{R} | 0.2 \le \epsilon \le 3.0  \} $
%incrementing by 0.1.
%For easily visualisation, Figs \ref{fig:RQA-IND} only show 3D RQA for 
%three recurrence thresholds $\epsilon=1, 2, 3$ 
Figs \ref{fig:RQA-IND} show the increase of recurrence threshold is 
associated to the increase of ENTR values in any of the levels 
of smoothness (see values of the ENTR bar).
Similarly, it can be noted that the increase of level of 
smoothness (from sg0 to sg2) is associated with both the increase 
of ENTR values and the smoothness of the 3D surfaces.
%One can also see in Figs \ref{fig:RQA-IND} that the maximum values of 
%ENTR for $\epsilon=1$ and sg0 are for embedding dimensions of 2 
%and then as $\epsilon$ increases the maximum values of ENTR are for
%an associated increase of in the embedding dimension.


% SECOND AMMENMEND
\subsection*{3D RQA ENTR}
It has been shown that one can select a default recurrence 
threshold of $\epsilon=1$ to represent the dynamical transitions of 
time series \cite{marwan2011}.
%Choosing an appropriate recurrence threshold is crucial so as to get 
%meaningful representations in RP and RQA, however, 
However, to show how that the selection of recurrence threshold affects RQA
values, 
we computed ENTR values of RQA metrics for different embedding parameters
$ \{ m \in \mathbb{R} | 1 \le m \le 10  \} $,
$  \{ \tau \in \mathbb{R} | 1 \le \tau \le 10  \} $ incrementing by one,
with the consideration of three recurrence thresholds $\epsilon=1, 2, 3$ 
and three levels of smoothness (sg0, sg1, sg2). 
%different recurrence thresholds 
%$\{ \epsilon \in \mathbb{R} | 0.2 \le \epsilon \le 3.0  \} $
%incrementing by 0.1.
%For easily visualisation, Figs \ref{fig:RQA-IND} only show 3D RQA for 
%three recurrence thresholds $\epsilon=1, 2, 3$ 
That being said, figures \ref{fig:RQA-IND} show the increase of recurrence threshold is 
associated to the increase of ENTR values in any of the levels 
of smoothness (see values of the ENTR bar).
Similarly, it can be noted that the increase of level of 
smoothness (from sg0 to sg2) is associated with both the increase 
of ENTR values and the smoothness of the 3D surfaces.
%One can also see in Figs \ref{fig:RQA-IND} that the maximum values of 
%ENTR for $\epsilon=1$ and sg0 are for embedding dimensions of 2 
%and then as $\epsilon$ increases the maximum values of ENTR are for
%an associated increase of in the embedding dimension.

%THIRD AMENMEND
\subsection*{3D RQA ENTR}
Marwan stated that the computation of an optimal recurrence threshold
$\epsilon$ is still an open question as it depends on a particular problem and question \cite{marwan2011}.
That said, recurrence threshold $\epsilon$ to study of dynamical transitions can be 
of little importance because of the relative change of the RQA measures 
does not depend too much on it in a certain range \cite{marwan2011}
%Choosing an appropriate recurrence threshold is crucial so as to get 
%meaningful representations in RP and RQA, however, 
However, to show how that the selection of recurrence threshold affects RQA
values, 
we computed ENTR values of RQA metrics for different embedding parameters
$ \{ m \in \mathbb{R} | 1 \le m \le 10  \} $,
$  \{ \tau \in \mathbb{R} | 1 \le \tau \le 10  \} $ incrementing by one,
with the consideration of three recurrence thresholds $\epsilon=1, 2, 3$ 
and three levels of smoothness (sg0, sg1, sg2). 
%different recurren



%%---------------------------------(FIGURE)-------------------------------------
\begin{figure}
\centering
\includegraphics[width=1.0\textwidth]{figures/rqa/output/rqa_epsilons}
    \caption
	[3D RQA ENTR values]{
	{\bf 3D RQA ENTR values.}
	RQA ENTR values are
	for different embedding parameters
	$ \{ m \in \mathbb{R} | 0 \le m \le 10  \} $,
	$ \{ \tau \in \mathbb{R} | 0 \le \tau \le 10  \} $
	incrementing by one and three recurrence thresholds $\epsilon=1, 2, 3$.
	RQA ENTR values were computed with data from $p03$, sensor HS01, with 
	a window size of 10-secs (500 samples).
	Code and data to reproduce the figure is available in \cite{srep2020}.
        }
    \label{fig:RQA-IND}
\end{figure}
%%---------------------------------(FIGURE)------------------------------------





\end{verbatim}

\textit{
LOG: \\ 
Tue 14 Apr 11:29:33 BST 2020\\
Sun 26 Apr 11:21:45 BST 2020\\
Thu 25 Jun 07:22:03 BST 2020\\
Thu 25 Jun 22:36:42 BST 2020
}
\\





\item  (pp. ) \\
ORIGINAL
\begin{verbatim}

\subsection*{Other effects on 3D RQA ENTR values}
%Zbilut et al. \cite{zbilut1992} established RQA metrics with the aim of 
%determining embedding parameters, their method consisted on creating 3D 
%surfaces with RQA metrics with an increase of embedding parameters 
%($m$ and $\tau$), then Zbilut et al. \cite{zbilut1992} explored 
%fluctuations and gradual changes in the 3D surfaces that provide information 
%about the embeddings. Much recently, Marwan et al. \cite{marwan2015} 
%created 3D surfaces for visual selection of not only embedding parameters 
%but also recurrence thresholds. Following same methodologies, we explored 
%the stability and robustness of RQA metrics (REC, DET, RATIO and ENTR)
%using 3D surfaces by an unitary increase of the pair embedding 
%parameters ($0 \ge m \le 10$, $0 \ge \tau \le 10$) and a decimal increase 
%of 0.1 for recurrence thresholds ($ 0.2 \ge \epsilon \le 3 $) 
%(Fig.~\ref{fig:topo_rqas}). 
We also computed 3D surfaces of RQA metrics for different sensors 
and different activities (Fig~\ref{fig:3dRQAENTR_sensoractivities}).
From the maximum values of ENTR (lateral bars), one can see 
a decrease of ENTR values for activities going from normal to 
faster velocity and from human sensor (HS01) to robot sensor (RS01) 
(See Fig~\ref{fig:3dRQAENTR_sensoractivities}). 
%%---------------------------------(FIGURE)-------------------------------------


	{\bf 3D surfaces of RQA ENTR metrics for sensors and activities.}
	RQA ENTR values with embedding parameters
	$[1,10]= \{ m \in \mathbb{R} | 0 \le m \le 10  \}$,
	$[1,10]= \{ \tau \in \mathbb{R} | 0 \le \tau \le 10  \}$
	for different activities and sensors. 
	RQA ENTR values are for $p03$, sg0 and window size of 10-secs (500 samples).
	Code and data to reproduce the figure is available in \cite{srep2020}.
       }

\end{verbatim}

AMENDED
\begin{verbatim}


%%\subsection*{Other effects on 3D RQA ENTR values}
%Zbilut et al. \cite{zbilut1992} established RQA metrics with the aim of 
%determining embedding parameters, their method consisted on creating 3D 
%surfaces with RQA metrics with an increase of embedding parameters 
%($m$ and $\tau$), then Zbilut et al. \cite{zbilut1992} explored 
%fluctuations and gradual changes in the 3D surfaces that provide information 
%about the embeddings. Much recently, Marwan et al. \cite{marwan2015} 
%created 3D surfaces for visual selection of not only embedding parameters 
%but also recurrence thresholds. Following same methodologies, we explored 
%the stability and robustness of RQA metrics (REC, DET, RATIO and ENTR)
%using 3D surfaces by an unitary increase of the pair embedding 
%parameters ($0 \ge m \le 10$, $0 \ge \tau \le 10$) and a decimal increase 
%of 0.1 for recurrence thresholds ($ 0.2 \ge \epsilon \le 3 $) 
%(Fig.~\ref{fig:topo_rqas}). 
Similarly, Figs~\ref{fig:3dRQAENTR_sensoractivities} show 3D surfaces of 
ENTR values for different sensors and different activities. 
Maximum values of ENTR (lateral bars) in Figs~\ref{fig:3dRQAENTR_sensoractivities} 
decrease for activities from normal (HN, VN) to faster (HF, VF) velocity and 
from human sensor (HS01) to robot sensor (RS01).


	{\bf 3D surfaces of RQA ENTR metrics for sensors and activities.}
	RQA ENTR values with embedding parameters
	$ \{ m \in \mathbb{R} | 0 \le m \le 10  \}$,
	$ \{ \tau \in \mathbb{R} | 0 \le \tau \le 10  \}$
	with $\epsilon = 1 $ for different activities and sensors. 
	RQA ENTR values are for $p03$, sg0 and window size of 10-secs (500 samples).
	Code and data to reproduce the figure is available in \cite{srep2020}.
 
\end{verbatim}

\textit{
LOG: \\ 
Tue 14 Apr 11:55:28 BST 2020
}
\\




\item  (pp. ) \\
ORIGINAL
\begin{verbatim}

% Create a "Supplementatry information"
% ~/srep2019/docs/supplementary-information/report
added: Sun 26 Apr 11:28:02 BST 2020
SORTED: Completion of new datasets, figures and report on Thu 25 Jun 00:50:15 BST 2020


However the manuscript should need a compresed version of such results to summarise
the following:
added: Thu 25 Jun 06:52:00 BST 2020


%%%%%%%%%%%%%%%%%%%%%%%%%%%%%%%%
% EPSILON, SMOTHNESS (DONE)
%X Three level of 
%X smoothness were computed for RQA metrics showing smoothed 3D surfaces and 
%X the level of smoothness increase (Fig.~\ref{fig:topo_smoothness}).

%%%%%%%%%%%%%%%%%%%%%%%%%%%%%%%%%%
%ACTIVITIES AND SENORS (DONE)

%%%%%%%%%%%%%%%%%%%%%%%%%%%%%%%%%
% COMPARE PARTICIPANTS 
%Similarly, 3D surfaces of RQA metrics were also computed for three 
%participants (Fig.~\ref{fig:topo_participants}).

%%%%%%%%%%%%%%%%%%%%%%%%%%%%%%%%%
% COMPARE WINDOWS
%RQA metrics are also affected by 
%the window length where for example four window lengths of 100, 250, 500 
%and 750 samples (Fig.~\ref{fig:topo_sensoractivities}). 




Figures~\ref{fig:3dRQAENTR_sensoractivities} show 3D surface plots of 
ENTR values for different sensors and different activities. 
It can be seen that maximum values of ENTR (lateral bars) in 
Figs~\ref{fig:3dRQAENTR_sensoractivities} decrease from normal (HN, VN) to faster (HF, VF) velocities 
and from human sensor (HS01) to robot sensor (RS01).
%%---------------------------------(FIGURE)-------------------------------------
\begin{figure}[ht]
\centering
\includegraphics[width=1.0\textwidth]{figures/rqa/output/rqa-sensors-activities}
    \caption{
	{\bf 3D surfaces plots of RQA ENTR values for sensors and activities.}
	RQA ENTR values with embedding parameters
	$ \{ m \in \mathbb{R} | 0 \le m \le 10  \}$,
	$ \{ \tau \in \mathbb{R} | 0 \le \tau \le 10  \}$
	with $\epsilon = 1 $ for different activities 
	(Horizontal Normal, Horizontal Faster, Vertical Normal, and Vertical Faster) and 
	sensors (Human Sensor 01 and Robot Sensor 01). 
	RQA ENTR values were computed from data of $p03$, sg0 and 
	window size of 10-secs (500 samples).
	Code and data to reproduce the figure is available in \cite{srep2020}.
       }
\label{fig:3dRQAENTR_sensoractivities}
\end{figure}
%%---------------------------------(FIGURE)-------------------------------------

Figures~\ref{fig:3dRQAENTR_participantsactivities}(A) show the similarity of 3D surface plots in 
relationship with four activities and three participants. 
It can be noted the subtle differences for each of the participants. 
However, 3D surface plots in figures~\ref{fig:3dRQAENTR_participantsactivities}(B) show 
little change for each participant as the data from a sensor attached to the robot. 
%%---------------------------------(FIGURE)-------------------------------------
\begin{figure}[ht]
\centering
\includegraphics[width=0.9\textwidth]{figures/rqa/output/rqa-participants}
    \caption{
	{\bf 3D surface plots of RQA ENTR value for different participants, activities and sensors}
	RQA ENTR values with embedding parameters
	$ \{ m \in \mathbb{R} | 0 \le m \le 10  \}$,
	$ \{ \tau \in \mathbb{R} | 0 \le \tau \le 10  \}$
	with $\epsilon = 1 $ for different activities 
	(Horizontal Normal, Horizontal Faster, Vertical Normal, and Vertical Faster) 
	three participants (p01, p02, and p03) and two main 
	classifications for (A) Human Sensor 01 and (B) Robot Sensor 01.
	RQA ENTR values were computed from data with sg0 and window size of 10-secs (500 samples).
	Code and data to reproduce the figure is available in \cite{srep2020}.
       }
\label{fig:3dRQAENTR_participantsactivities}
\end{figure}
%%---------------------------------(FIGURE)-------------------------------------


Figures \ref{fig:3dRQAENTR_windowsactivities} show the effect on the 3D surface plots
as the window length of the time series increase (w100 (2-sec), w250 (5-sec), w500 (10-sec) and w750 (15-sec)).
It can be noted that the increase of number of samples improves the quality 
of the surface plots where 100 samples roughly capture the dynamics of the activity
and 250, 500 and 750 samples show similar representation of RQA Entropy values 
(being the surface plot with 750 samples the best test case).
%%---------------------------------(FIGURE)-------------------------------------
\begin{figure}[ht]
\centering
\includegraphics[width=1.0\textwidth]{figures/rqa/output/rqa-windows}
    \caption{
	{\bf 3D surface plots of RQA ENTR values for windows and activities.}
	RQA ENTR values for embedding parameters
	$ \{ m \in \mathbb{R} | 0 \le m \le 10  \}$,
	$ \{ \tau \in \mathbb{R} | 0 \le \tau \le 10  \}$, 
	$\epsilon = 1 $ for different 
	windows size (e.g. w100 (100 samples), w250 (250 samples),
	w500 (500 samples) and w750 (750 samples)).
	RQA ENTR values were computed from data of $p01$ and sg0.
	Code and data to reproduce the figure is available in \cite{srep2020}.
       }
\label{fig:3dRQAENTR_windowsactivities}
\end{figure}
%%---------------------------------(FIGURE)-------------------------------------








Mon 29 Jun 22:35:05 BST 2020
For instance, 
Figures \ref{fig:RQA-IND} show the increase of recurrence threshold is 
associated to the increase of ENTR values in any of the levels 
of smoothness (see values of the ENTR bar).
Similarly, it can be noted that the increase of level of 
smoothness (sg0, sg1 and sg2) is associated with the increase 
of ENTR values of the 3D surface plots.
%%---------------------------------(FIGURE)-------------------------------------
\begin{figure}
\centering
\includegraphics[width=1.0\textwidth]{figures/rqa/output/rqa-epsilons}
    \caption
	[3D surface plots of RQA ENTR values]{
	{\bf 
	3D surface plots of RQA ENTR values for different recurrence threshold and smoothness levels.}
	RQA ENTR values are
	for embedding parameters
	$ \{ m \in \mathbb{R} | 0 \le m \le 10  \} $,
	$ \{ \tau \in \mathbb{R} | 0 \le \tau \le 10  \} $
	incrementing by one and three recurrence thresholds $\epsilon=1, 2, 3$.
	RQA ENTR values were computed with data from $p03$, sensor HS01, with 
	a window size of 10-secs (500 samples).
	Code and data to reproduce the figure is available in \cite{srep2020}.
        }
    \label{fig:RQA-IND}
\end{figure}
%%---------------------------------(FIGURE)------------------------------------
Figures~\ref{fig:3dRQAENTR_sensoractivities} show 3D surface plots of 
ENTR values for different sensors and different activities.
Figs~\ref{fig:3dRQAENTR_sensoractivities} show that RQA ENTR (lateral bars) 
decrease from normal (HN, VN) to faster (HF, VF) velocities from both 
human sensor (HS01) to robot sensor (RS01).
Also, it can be noted that RQA ENTR values decrease from human sensor (HS01) 
to robot sensor (RS01).
%%---------------------------------(FIGURE)-------------------------------------
\begin{figure}[ht]
\centering
\includegraphics[width=1.0\textwidth]{figures/rqa/output/rqa-sensors-activities}
    \caption{
	{\bf 3D surface plots of RQA ENTR values for different sensors and activities.}
	RQA ENTR values are for embedding parameters
	$ \{ m \in \mathbb{R} | 0 \le m \le 10  \}$,
	$ \{ \tau \in \mathbb{R} | 0 \le \tau \le 10  \}$
	with $\epsilon = 1 $ considering four activities 
	Horizontal Normal (HN), Horizontal Faster(HF), Vertical Normal(VN), and 
	Vertical Faster (VF) and sensors Human Sensor 01 (HS01) and 
	Robot Sensor (RS01).
	RQA ENTR values were computed from data of $p03$, sg0 and 
	window size of 10-secs (500 samples).
	Code and data to reproduce the figure is available in \cite{srep2020}.
       }
\label{fig:3dRQAENTR_sensoractivities}
\end{figure}
%%---------------------------------(FIGURE)-------------------------------------
Figures~\ref{fig:3dRQAENTR_participantsactivities} show the similarity of 3D surface plots in 
relationship with three participants (p01, p02 and p03)
performing four activities (HN, HF, VN, and VF) and two sensors (HS01, RS01).
Figures~\ref{fig:3dRQAENTR_participantsactivities}(A) 
present subtle differences for each of the participants (see RQA ENTR bar). 
Figures~\ref{fig:3dRQAENTR_participantsactivities}(B) show 
less change of RQA ENTR values for each participant as the data is 
from a sensor attached to the robot. 
Also, from data of sensors attached to the robot, it can be noted 
that changes are more notable for faster movements than normal movements. 
%%---------------------------------(FIGURE)-------------------------------------
\begin{figure}[ht]
\centering
\includegraphics[width=0.9\textwidth]{figures/rqa/output/rqa-participants}
    \caption{
	{\bf 3D surface plots of RQA ENTR values for different participants, activities and sensors.}
	RQA ENTR values are for participants (p01, p02, and p03) 
	in the categories of 
	(A) Human Sensor 01 (HS01) and 
	(B) Robot Sensor 01 (RS01)
	considering embedding parameters
	$ \{ m \in \mathbb{R} | 0 \le m \le 10  \}$,
	$ \{ \tau \in \mathbb{R} | 0 \le \tau \le 10  \}$
	with $\epsilon = 1$ and four activities 
	Horizontal Normal (HN), Horizontal Faster(HF), Vertical Normal(VN), and 
	Vertical Faster (VF).
	RQA ENTR values were computed from data of sg0 and window size of 10-secs (500 samples).
	Code and data to reproduce the figure is available in \cite{srep2020}.
       }
\label{fig:3dRQAENTR_participantsactivities}
\end{figure}
%%---------------------------------(FIGURE)-------------------------------------
Figures \ref{fig:3dRQAENTR_windowsactivities} show the effect of the increase 
of window length of the time series (w100 (2-sec), w250 (5-sec), w500 (10-sec) and w750 (15-sec))
on the 3D surface plots. 
It can be noted that the increase of number of samples improves the quality 
of the surface plots where 100 samples poorly capture the dynamics of each activity.
However as the window length increases from 250, 500 to 750 samples, 
3D surface plots show a similar representation of RQA ENTR values 
(being the surface plot with 750 samples the best test case).
%%---------------------------------(FIGURE)-------------------------------------
\begin{figure}[ht]
\centering
\includegraphics[width=1.0\textwidth]{figures/rqa/output/rqa-windows}
    \caption{
	{\bf 3D surface plots of RQA ENTR values for different windows lengths and activities.}
	RQA ENTR values are for embedding parameters
	$ \{ m \in \mathbb{R} | 0 \le m \le 10  \}$,
	$ \{ \tau \in \mathbb{R} | 0 \le \tau \le 10  \}$, 
	with $\epsilon = 1 $ considering four 
	windows lengths (e.g., w100 (100 samples), w250 (250 samples),
	w500 (500 samples) and w750 (750 samples)) and
	four activities 
	Horizontal Normal (HN), Horizontal Faster(HF), Vertical Normal(VN), and 
	Vertical Faster (VF).
	RQA ENTR values were computed from data of $p01$ and sg0.
	Code and data to reproduce the figure is available in \cite{srep2020}.
       }
\label{fig:3dRQAENTR_windowsactivities}
\end{figure}
%%---------------------------------(FIGURE)-------------------------------------





\end{verbatim}

AMENDED
\begin{verbatim}

\end{verbatim}

\textit{
LOG: \\ 
Thu 25 Jun 06:48:57 BST 2020 \\
Fri 26 Jun 03:31:54 BST 2020 \\
Mon 29 Jun 22:35:15 BST 2020
}
\\





\item  (pp. ) \\
ORIGINAL
\begin{verbatim}

\end{verbatim}

AMENDED
\begin{verbatim}

\end{verbatim}

\textit{
LOG: \\ 
}
\\





\item  (pp. ) \\
ORIGINAL
\begin{verbatim}

\end{verbatim}

AMENDED
\begin{verbatim}

\end{verbatim}

\textit{
LOG: \\ 
}
\\





\item  (pp. ) \\
ORIGINAL
\begin{verbatim}

\end{verbatim}

AMENDED
\begin{verbatim}

\end{verbatim}

\textit{
LOG: \\ 
}
\\







\end{enumerate}






%%%%%%%%%%%%%%%%%%%%%%%%%%%%%%%
\subsection{Discussion}
\begin{enumerate}



\item  (pp. ) 

ORIGINAL
\begin{verbatim}



Tue 30 Jun 22:28:39 BST 2020 
%The Discussion should be succinct and must not contain subheadings.

%Zbilut et al. \cite{zbilut1992} established RQA metrics with the aim of 
%determining embedding parameters, their method consisted on creating 3D 
%surfaces with RQA metrics with an increase of embedding parameters 
%($m$ and $\tau$), then Zbilut et al. \cite{zbilut1992} explored 
%fluctuations and gradual changes in the 3D surfaces that provide information 
%about the embeddings. Much recently, Marwan et al. \cite{marwan2015} 
%created 3D surfaces for visual selection of not only embedding parameters 
%but also recurrence thresholds. Following same methodologies, we explored 
%the stability and robustness of RQA metrics (REC, DET, RATIO and ENTR)
%using 3D surfaces by an unitary increase of the pair embedding 
%parameters ($0 \ge m \le 10$, $0 \ge \tau \le 10$) and a decimal increase 
%of 0.1 for recurrence thresholds ($ 0.2 \ge \epsilon \le 3 $) 
%(Fig.~\ref{fig:topo_rqas}).


It is evident that time series from different sources 
(participants, movements, axis type, window length or levels of smoothness) 
presents visual differences for embedding parameters and therefore for RRSs. 
For which, the selection of embedding parameters was our first challenge 
where we computed embedding parameters for each time series 
(Fig \label{fig:caoami}) and then computed a sample mean over 
all time series in order to get two embedding parameters 
to compute all RRSs (Figs \label{fig:rss_aHw10} and \label{fig:rss_aVw10}). 
%with its corresponded type of movement. 
Then we found that the quantification of variability with regard to 
the shape of the trajectories in RSSs requires more investigation 
since our original proposed method base on euclidean metric failed 
to quantify those trajectories. Specially, for trajectories which 
were not well unfolded. 
With that in mind, we apply 
Recurrence Quantification Analysis (RQA) metrics 
(REC, DET, RATIO and ENTR) in order to avoid any subjective interpretations 
or personal bias with regard to the representation of the trajectories in RSSs.


\subsection*{RQA metrics with fixed parameters}
Considering that RQA metrics were computed with fixed embedding parameters 
($m=6$ and $\tau=8$) and recurrence thresholds ($\epsilon=1$), we found 
the following. REC values, which represents the \% of black points in the RPs, 
were more affected with and increase in normal speed movements (HN and VN) 
than faster movements (HF and VF) for the sensor attached to the participants 
(HS01). Such decrease of REC values from normal speed to faster speed 
movements is also presented in data from sensor attached to the robot (RS01), 
and little can be said with regard to the dynamics of the time series coming 
from RS01 (Fig \ref{fig:RQABP}A).
Similarly, DET values, representing predictability and 
organisation in the RPs, present little variation in the any of the time 
series where little can be said (Fig \ref{fig:RQABP}B).
In contrast, RATIO values, which represent 
dynamic transitions, were more variable for faster movements (HF and VF) 
than normal speed movements (HN and VN) with sensors attached to the 
participants (HS01). For data coming from sensors attached to the robot 
(RS01), RATIO values from horizontal movements (HN, HF) appear to vary 
more than values coming from vertical movmentes (VN, VF) 
(Fig \ref{fig:RQABP}C).
With that, it can be said that RATIO values can represent a bit better
than REC or DET metrics for the variability of imitation activities in 
each of the conditions for time series.
Similarly, ENTR values for HN were higher than values for HF
and ENTR values varied more for sensor attached to participants 
than ENTR values for sensors of the robot. It is evidently that 
the higher the entropy the more complex the dynamics are, 
however, ENTR values for HN appear a bit higher than HF values, 
for which we believe this happens because of the structure the time series
which appear more complex for HN than  HF movements which presented a 
more consistence repetition (Fig \ref{fig:RQABP}D).


Additionally, we observed that some RQA metrics are affected by the 
smoothness of data. 
%For which, we also explored the effect of smoothness of raw-normalised data 
%where, 
For instance, REC and DET values were not completely affected by the 
smoothness of time-series since these RQA values seemed to be constants. 
However, for RATIO values, 
the effect of smoothness can be noticed with a slight decrease of amplitude 
in any of the time series conditions which is also presented with ENTR values.


\subsection*{RQA metrics with different parameters}
Iwansky et al. \cite{iwanski1998} stated that patterns in RPs and 
metrics for RQA are independent of embedding dimension parameters, 
however, that is not the case when using 
different recurrence thresholds. Such changes of recurrence threshold values 
can modify the patterns in RPs and therefore the values of RQA metrics.
We therefore computed 3D surfaces to explore the sensibility and robustness of 
embedding parameters and recurrence threshold in RQA  metrics. Following the 
same methodology of computing 3D surfaces, we also considered variation of 
window length size to present RQA metrics dependencies with embedding 
parameters, recurrence thresholds and window length size.




\end{verbatim}
AMENDED

\begin{verbatim}


Wed  1 Jul 00:33:42 BST 2020 
%*******************************************************************************
%*******************************************************************************
%*******************************************************************************
\section*{Discussion}
Time series from different sources  
(e.g., participants, movements, axis type) as well as
different characteristis (e.g., sample rate, window length or levels of smoothness) 
result in different embedding parameters and therefore different 
values for recurrence plots and recuerence quantification analysis metrics.
That said, while embedding parameters for individual
time series were succesfully computed, it has been shown that the quantification 
of variability with regard to the shape of the trajectories in 
reconstructed state spaces require more investigation
as our initial results based on euclidean distances 
in the trajectories failed to quantify trajectories 
which were not well unfolded. 
Then, the use of Recurrence Quantification Analysis (RQA) metrics 
(e.g., REC, DET, RATIO and ENTR) helped to quantify 
the variablity of different sources of time series, to then found that 
ENTR metric along with different recurrence thresholds showed 
the variation of different sources of time series.





Thu  2 Jul 22:26:10 BST 2020
%*******************************************************************************
\section*{Discussion}
Time series from different sources  
(e.g., participants, movements, axis type) as well as
different characteristics (e.g., sample rate, window length or levels of smoothness) 
result in different embedding parameters and therefore different 
values for recurrence plots and recurrence quantification analysis metrics.
That said, while embedding parameters for individual
time series were successfully computed, the quantification 
of variability with regard to the shape of the trajectories in 
reconstructed state spaces require more investigation
as our initial, approach based on euclidean distances on the trajectories, 
failed to quantify trajectories which were not well unfolded. 
To then found out that Recurrence Plots along with 
Recurrence Quantification Analysis (RQA) metrics (e.g., REC, DET, RATIO and ENTR) 
and its variation of embedding parameters and recurrence thresholds 
helped to quantify the variability of different sources of time series.
However, further investigation is required to be done in order to 
have better intuition and meaningful interpretation of nonlinear 
analysis such as to find a right 
balance among (i) the level of smoothness of the signal, 
(ii) the selection of recurrence thresholds and (iii) 
the range of embedding parameters. 

\


\end{verbatim}

	\textit{
	LOG: \\
	Tue 30 Jun 22:28:39 BST 2020 \\
	Wed  1 Jul 00:33:42 BST 2020 \\
	Thu  2 Jul 22:26:16 BST 2020 \\
	}
	\\




\end{enumerate}






\subsection{Conclusions}

\begin{enumerate}


\item  (pp. ) 

ORIGINAL
\begin{verbatim}

\section*{Conclusions}
Generally, we can conclude that using a different level of 
smoothness for time series help us to visualise and to quantify 
the variation of movements 
between participants using RSSs, RPs and RQA. 
It is important to mention that some RQA's metrics 
(e.g. DET and ENTR) are more robust to the effect of 
smoothness of time series.
Choice of RQA metrics will be depend on 
what to quantify, for instance, one can compute 
predicability, organisation,  
dynamics transitions, or complexity and determinism of the 
input time series. However, RQA metrics show certain constrains with 
regard to activity type, window length and structure of the time series.
For instance, RATIO and ENTR are helpful to distinguish 
differences in any of the categories of the time series (sensor, activity, 
level of smoothness and number of participant), however time series
data from the sensor attached to the robot seemed to have 
little variations between as humanoid degrees of freedom 
did not allow it to move with a wide range of variability. 

Additionally, we can point out that even though our experiment is 
limited to twenty healthy right-handed participants of a range age 
of mean 19.8 SD=1.39, RQA metrics show the potential to quantify 
human movement variability in the context of 
human-humanoid imitation activities.
%%%%%%%%%%%%%%%%%%%%%%%%%%%%%%%%%%%%%%%%%%%%
%	You need a sentence or two here
%	to explain why this is claimed.
%%%%%%%%%%%%%%%%%%%%%%%%%%%%%%%%%%%%%%%%%%%%
	
%Additionally, we observed that some RQA metrics are affected by the 
%smoothness of data. 
%%For which, we also explored the effect of smoothness of raw-normalised data 
%%where, 
%For instance, REC and DET values were not completely affected by the 
%smoothness of time-series since these RQA values seemed to be constants. 
%However, for RATIO values, 
%the effect of smoothness can be noticed with a slight decrease of amplitude 
%in any of the time series conditions which is also presented with ENTR values.



%With that in mind, we conclude that 
%quantification of human-humanoid imitation activities is possible for 
%participants of different ages, state of health and anthropomorphic features.


%Such understanding and measurement of movement variability using
%cheap wearable inertial sensors lead us to have a more intuitive selection of parameters
%to reconstruct the state spaces and to create meaningful interpretations
%of the recurrence plots and the results of the metrics with recurrence quantification 
%analysis. 
%


%
%However, we believe that further investigation is required to find the 
%right balance between the level of smoothness of the signal
%and defining what is the aim of RQA where 
%its representations using RSS, RP and RQA.
%Specially, where the level of smoothness does not affect 
%the variation of each of the movements quantification. 
%%We can also conclude that finding the right balance between 
%%smoothness and the raw data to capture movement variability is 
%%a still a problem that has many avenues for exploration.
%

%Hence, variation of recurrence thresholds and embedding parameters 
%has been presented to show the relationships to different datasets 
%(participants, activities, windows lengths and sensors).





\end{verbatim}
AMENDED
\begin{verbatim}

Thu  2 Jul 00:15:27 BST 2020
\section*{Conclusions}
This work allows to conclude that the choice of 
nonlinear analysis tool (e.g., RSSs, RPs, RQA metrics) will depends 
on what one would like to quantify on the time-series 
data (e.g., predicability, organisation, dynamics transitions, 
or complexity and determinism).
Then, time-series data characteristics 
(e.g., window size length, level of smoothness) plays an important role
as well on the results that nonlinear analysis tool can provide.
Similarly, the results of the nonlinear analysis tools
are associated with the structure of the time-series data 
(e.g. frequency, amplitude), the position of the sensor 
and activity performed by either a robot or human being
(as degrees of freedom from the humanoid are far less than
human movement). 
That said, it is unquestionable that the use of different 
characteristics of the time-series data 
(e.g., sensor, activity, level of smoothness and participants)
has help us to visualise and to quantify 
with nonlinear tools the variation of 
movements of our human-humanoid activities.
However, with some limitation of the nonlinear tools such as 
RSSs cannot unfold trajectories, RPs 
are dependent on recurrence thresholds
and RQA metrics show certain constrains with 
regard to activity type, window length and structure of the time series.
Specifically, DET values appear to be constants across sensors, 
activities and levels of smoothness, 
whereas REC and RATIO, as function of REC, values 
show variation for certain sensors and movements. To then 
found that RQA ENTR values with recurrence thresholds  
were appropriate to quantify 
the different changes and variations of the characteristics of 
time-series data.

Further investigation is required to be done in order to 
have more intuition and meaningful interpretation of nonlinear 
analysis tools as well as to find a right 
balance among (i) the level of smoothness of the signal, 
(ii) the selection of recurrence thresholds and (iii) 
the range of embedding parameters. 
Nonetheless, this work provides a good starting 
point to conclude that Shannon Entropy is an approach that can
lead to interesting results on the quantification of human-humanoid imitation activities 
for participants of different ages, state of health and anthropomorphic features.





Thu  2 Jul 22:41:43 BST 2020
\section*{Conclusions}
This work allows to conclude that the choice of 
nonlinear analysis tool (e.g., RSSs, RPs, RQA metrics) will depend 
on what one would like to quantify on the time-series 
data (e.g., predicability, organisation, dynamics transitions, 
or complexity and determinism).
Then, time-series data characteristics 
(e.g., window size length, level of smoothness) plays an important role
as well on the results that nonlinear analysis tool can provide.
Similarly, the results of the nonlinear analysis tools
are associated with the structure of the time-series data 
(e.g. frequency, amplitude), the position of the sensor 
and activity performed by either a robot or human being
(as degrees of freedom from the humanoid are far less than
human movement). 
That said, it has been shown that the use of different 
characteristics of the time-series data 
(e.g., sensor, activity, level of smoothness and participants)
has help us to visualise and to quantify 
with nonlinear tools the variation of 
movements of, in this work, human-humanoid activities.
However, some limitation of nonlinear tools are related 
to the computation of different parameters 
(e.g., recurrence thresholds, embedding parameters)
that reflect the dynamics of individual characteristics 
of activity type, window length and structure of the time series.
Specifically, the example of DET values which appear 
to be constants across sensors, activities and levels of smoothness, 
whereas REC and RATIO, as function of REC, values 
show variation for certain sensors and movements. 
To then found out that RQA ENTR values with different recurrence 
thresholds were appropriate to quantify 
the different changes and variations of the characteristics of 
time-series data.
Therefore, we can conclude that this work provides a good starting 
point and reference to the use of Shannon Entropy to quantify 
human-humanoid imitation activities 
that can then lead to interesting results on the quantification 
of movement variability of participants with different ages, 
state of health and anthropomorphic features.


\end{verbatim}

	\textit{
	LOG:\\ 
	Thu  2 Jul 00:15:27 BST 2020 \\
	Thu  2 Jul 22:41:48 BST 2020 \\
	}
	\\




\end{enumerate}





\subsection{Future work}

\begin{enumerate}


\item  (pp. ) 

ORIGINAL
\begin{verbatim}

\section*{Future Work}

\subsection*{Inertial Sensors}
To better understand the signals collected through 
inertial sensors in the context of human-humanoid interaction, 
we could apply derivates to the acceleration data. 
We can then explore the jerkiness of movements and therefore 
the nature of arm movements which typically have 
minimum jerk \cite{flash1985}, its relationship with different body 
parts \cite{devries1982, mori2012} or the application of higher derivatives 
of displacement with respect time such as snap, crackle and pop \cite{eager2016}.

\subsection*{RQA ENTR}
Having shown that RQA ENTR metrics are robust against different data
time series, we believe that further investigation is required. 
For example, Marwan et al. \cite{marwan2007, marwan2015} reviewed 
different aspects to compute RPs using different criteria for neighbours, 
different norms ($L_{1-norm}$, $L_{2-norm}$, or $L_{\infty-norm}$ ) or 
different methods to select the recurrence threshold $\epsilon$, 
such as using certain percentage of the signal \cite{letellier2006}, 
the amount of noise or using a factor based on the standard deviation 
of the observational noise among many others \cite{marwan2007}.



\end{verbatim}
AMENDED
\begin{verbatim}

Future work is not necessarily part of the manuscript in the guidelines 
for which I leave it out for coming publications
\end{verbatim}

	\textit{
	LOG:\\ 
	Wed  1 Jul 00:39:16 BST 2020
	}
	\\




\end{enumerate}






\subsection{Abstract}

\begin{enumerate}


\item  (pp. ) 

ORIGINAL
\begin{verbatim}

\begin{abstract}
Human movement variability occurs in motor performance 
across multiple repetitions of a task and such behaviour is an
inherent feature within and between each persons' movement. Quantifying
movement variability is still an open problem, particularly when traditional
methods in time domain and frequency domain fail to detect tiny modulations in
frequency or phase for time series. 
For this work, we hence investigate methodologies from nonlinear dynamics such as 
reconstructed state space (RSS), uniform time-delay embedding (UTDE), 
recurrence plots (RPs) and recurrence quantification analysis (RQA) metrics. 
Particularly, we are interested in the weaknesses and robustness 
of nonlinear dynamics tools from raw and post-processed data of 
wearable inertial sensors (IMUs). 
In the reported experiment, twenty right-handed healthy participants 
imitated simple vertical and horizontal arm movements in normal 
and faster speed from an humanoid robot.
With four window lengths and three levels of smoothed time series, 
we found visual differences in the patterns of RSSs and RPs. 
Then with the use of RQA metrics, we find out that the type of 
movements and the level of smoothness affects those metrics. 
Specifically, entropy values from RQA were well distributed and 
presented variation in all the conditions for time series. 
This work have then the potential to enhance the development of 
better diagnostic tools for various applications in rehabilitation, 
sport science or for new forms of human-robot interaction.
\end{abstract}
%Number of words: 226





\end{verbatim}
AMENDED
\begin{verbatim}


Thu  2 Jul 00:37:16 BST 2020
\begin{abstract}
Human movement variability occurs in motor performance 
across multiple repetitions of a task and such behaviour is an
inherent feature within and between each persons' movement. Quantifying
movement variability is still an open problem, particularly 
when methods in time domain, frequency domain or nonlinear dynamics 
can break down due to the real-world time-series datasets. 
For this work, we hence investigate methodologies from nonlinear dynamics such as 
reconstructed state space (RSS), uniform time-delay embedding (UTDE), 
recurrence plots (RPs) and recurrence quantification analysis (RQA) metrics
with real-world time-series data.
Particularly, we are interested in the weaknesses and robustness 
of nonlinear dynamics tools of raw and post-processed data of 
wearable inertial sensors. 
That said, twenty right-handed healthy participants 
imitated simple vertical and horizontal arm movements in normal 
and faster velocity from an humanoid robot in order to have 
four window lengths and three levels of smoothed time-series data,
to then found visual differences in the patterns with RSSs and RPs. 
To then find out that the the use of RQA metrics
can quantify activities, types of sensors, windows lenghts 
and the level of smoothness. 
Specifically, we can conclude that Shannon Entropy can lead to 
interesting results on the quantification of human-humanoid imitation activities
for participants of different ages, state of health and anthropomorphic features.
to then enhance the development of 
better diagnostic tools for various applications in rehabilitation, 
sport science or for new forms of human-robot interaction.
\end{abstract}



Thu  2 Jul 22:54:44 BST 2020
\begin{abstract}
Human movement variability occurs in motor performance 
across multiple repetitions of a task and such behaviour is an
inherent feature within and between each persons' movement. Quantifying
movement variability is still an open problem, particularly 
when methods in time domain, frequency domain or nonlinear dynamics 
can break down due to the real-world time-series datasets. 
For this work, we therefore investigate nonlinear dynamics methods 
such as reconstructed state space (RSSs), uniform time-delay embedding, 
recurrence plots (RPs) and recurrence quantification analysis metrics (RQAs)
with real-world time-series data.
Particularly, we are interested in the weaknesses and robustness 
of nonlinear dynamics tools with the use of raw and post-processed 
data of wearable inertial sensors. 
That said, twenty right-handed healthy participants 
imitated simple vertical and horizontal arm movements in normal 
and faster velocity from an humanoid robot in order to have 
four window lengths and three levels of smoothed time-series data,
to then found visual differences in the patterns with RSSs and RPs. 
We then found out that RQA metrics
can quantify activities, types of sensors, windows lenghts 
and level of smoothness. 
Specifically, we can conclude that Shannon Entropy can lead to 
interesting results on the quantification of 
movement variablity for participants of different ages, 
state of health and anthropomorphic features.
to then potentially enhance the development of 
better diagnostic tools for applications in rehabilitation, 
sport science or for new forms of human-robot interaction.
\end{abstract}


Thu  2 Jul 23:03:20 BST 2020
\begin{abstract}
Human movement variability occurs in motor performance 
across multiple repetitions of a task and such behaviour is an
inherent feature within and between each persons' movement. Quantifying
movement variability is still an open problem, particularly 
when methods in time domain, frequency domain or nonlinear dynamics 
can break down due to the real-world time-series datasets. 
For this work, we therefore investigate nonlinear dynamics methods 
such as reconstructed state space (RSSs), uniform time-delay embedding, 
recurrence plots (RPs) and recurrence quantification analysis metrics (RQAs)
with real-world time-series data.
Particularly, we are interested in the weaknesses and robustness 
of nonlinear dynamics tools with the use of raw and post-processed 
data of wearable inertial sensors. 
That said, twenty right-handed healthy participants 
imitated simple vertical and horizontal arm movements in normal 
and faster velocity from an humanoid robot in order to have 
four window lengths and three levels of smoothed time-series data,
to then found visual differences in the patterns with RSSs and RPs
and particulatly the computed differences with RQA metrics
that help us to quantify activities, types of sensors, windows lenghts 
and level of smoothness. 
Specifically, we can conclude that RQA ENTR, Shannon Entropy, 
can lead to interesting results on the quantification of 
movement variablity for participants of different ages, 
state of health and anthropomorphic features to 
then enhance the development of 
better diagnostic tools for applications in rehabilitation, 
sport science or for new forms of human-humanoid interaction.
\end{abstract}




\end{verbatim}

	\textit{
	LOG:\\ 
	Thu  2 Jul 00:37:21 BST 2020\\
	Thu  2 Jul 22:54:53 BST 2020\\
	Thu  2 Jul 23:03:20 BST 2020
	}
	\\




\end{enumerate}



\subsection{Title}

\begin{enumerate}

\item  (pp. ) 

ORIGINAL
\begin{verbatim}
\title{Recurrent Quantification Analysis of Human-Humanoid Interaction Activities}
\end{verbatim}

AMEND 
\begin{verbatim}
Thu  2 Jul 22:44:53 BST 2020
\title{Recurrent Quantification Analysis of Movement Variability in Human-Humanoid Interaction Activities}
\end{verbatim}




	\textit{
	LOG:\\ 
	Thu  2 Jul 22:44:53 BST 2020
	}
	\\




\end{enumerate}





\end{document}

